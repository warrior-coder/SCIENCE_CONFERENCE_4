\documentclass[a4paper,10pt,twoside]{article}
\usepackage{conf_template}
\setcounter{section}{0}
\setcounter{figure}{0}
\setcounter{table}{0}
\setcounter{equation}{0}
\setcounter{secnumdepth}{1}
\setcounter{secnumdepth}{1}
\begin{document}
% ================================= Начало Вашей статьи. Всё, что выше - не изменять ! ================================
%

\authors {% авторы статьи, между инициалами - пробел
А.~С.~Гудков, А.~А.~Лысенко, Е.~А.~Семёнов
}%конец списка авторов
%
\topic{%начало заголовка статьи
АВТОМАТИЗИРОВАННАЯ СИСТЕРМА РАСЧЕТА ЭЛЕКТРИЧЕСКИХ ЦЕПЕЙ ПОСТОЯННОГО И ПЕРЕМЕННОГО ТОКОВ
}%конец заголовка статьи
%
% ------------- Аннотация -----------------------
\annotation{%% начало текста аннотации
Рассматривается подход к автоматизации расчета электрических цепей и реализация соответствующего программного обеспечения на языке программирования C++.
}% конец текста аннотации
\begin{multicols}{2}
% =================================== Начало основного текста ===========================================================
% ===== Первый раздел статьи =================================================
%
\section*{% заголовок 1 раздела статьи
Введение}% начало текста 1 раздела статьи
В процессе изучения дисциплины «Теория электрических цепей» авторами был предложен универсальный алгоритм расчета электрических цепей как постоянного, так и переменного токов, и принято решение о его реализации в виде ПО. На основе сгенерированных задач для студентов, изучающих данную дисциплину, с помощью ПО необходимо подготовить ответы к соответствующим задачам для преподавателей.
% конец текста 1 раздела статьи
% ===== Второй раздел статьи ===================================================
%
\section{% заголовок 2 раздела статьи
Описание алгоритма
}\content{% начало текста 2 раздела статьи
Для того чтобы программа могла работать с электрической цепью, ее следует представить в виде математической модели, поскольку машина выполняет только арифметические действия. Было принято вычислять токи матричным методом узловых потенциалов, поэтому в данной задаче математическая модель представляется матрицами двух видов:

\textbf{--} топологические матрицы (характеризуют структурные особенности цепи);

\textbf{--} компонентные матрицы (отражают характеристики компонентов цепи).

Для составления топологических матриц исходная цепь представляется в виде ориентированного графа, направления ребер которого совпадают с направлениями токов в ветвях, а вершины с узлами цепи. На его основе формируется топологическая узловая матрица. За базисный узел принимается последняя сгенерированная вершина графа. На этом этапе заметно преимущество данного подхода перед методом контурных токов -- отсутствие необходимости искать независимые контуры цепи. 

Учитывая то, что каждая ветвь цепи может быть представлена обобщенной, формируются компонентные матрицы-столбцы c исходными данными цепи для \textit{E}, \textit{J}, \textit{R}, и диагональная \textit{RD}, размерности которых равны количеству ветвей цепи.
}% конец текста 2 раздела статьи
% ====== Третий раздел статьи ===================================================
%
\section{% заголовок 3 раздела статьи
Составление программы
}\content{% начало текста 3 раздела статьи
Исходные данные электрической цепи хранятся в виде таблицы в текстовом документе. Для переменного тока таблица расширяется реактивными элементами и аргументами источников энергии. По дынным из таблицы удобно формировать граф цепи, т.к. каждая стока таблицы отражает конкретную ветвь цепи. 

Для реализации логики матричного исчисления в программе были разработаны классы \textit{MatrixF} и \textit{MatrixC} (унаследованные от базового \textit{Matrix<T>} с шаблонными параметрами \textit{Float} и \textit{Complex}). При расчете цепи переменного тока вводится комплексный параметр -- класс \textit{Compex}, позволяющий выполнять преобразования над комплексными числами в алгебраической и экспоненциальной формах.

В классе-интерфейсе электрической цепи \textit{Circuit} задаются обобщенные матричные уравнения метода узловых потенциалов. Возвращаемым значением метода \textit{СalculateCircuit()} является результат расчета -- матрица-столбец \textit{IR} токов в сопротивлениях ветвей. Его наследуют классы \textit{CircuitDC} и \textit{CircuitAC}, несущие матмодели цепей постоянного и переменного токов.
}% конец текста 3 раздела статьи

\section{Выводы}
В ходе работы был рассмотрен подход к автоматизации расчета электрических цепей и реализована соответствующая программа, позволяющая получать максимально точные значения токов для электрических цепей c любой топологией. ПО полезно как для студентов, изучающих ТЭЦ, так и для преподавателей, составляющих задачи по этой дисциплине. 
%
% ***** Пример добавления маркированного списка ************************************
%
\references{
}\ListReferences{% начало списка литературы
\item Артым~А.~Д. Новый метод расчета процессов в электрических цепях / А.~Д.~Артым, В.~А.~Филин, К.~Ж.~Есполов // СПб.: «Элмор»,~-- 2001.~-- 192 с.
}% ---- конец списка литературы
%

\end{multicols}
% ====== Пример оформления данных об авторах ==============================================================
%
\authorFIO{Гудков Алексей Сергеевич}
\authorAbout{
студент 2 курса кафедры информационных технологий автоматизированных систем БГУИР, gudkov\_fitu@mail.ru.
}

\authorFIO{Лысенко Антон Александрович}
\authorAbout{
студент 2 курса кафедры информационных технологий автоматизированных систем БГУИР, toshka.lysenko.15@gmail.com.
}

\authorFIO{Семёнов Егор Александрович}
\authorAbout{
студент 2 курса кафедры информационных технологий автоматизированных систем БГУИР, egor123semenov@gmail.com.
}


\authorFIO{Научный руководитель: Шилин Леонид Юрьевич}
\authorAbout{
декан факультета информационных технологий и управления БГУИР, доктор технических наук, профессор, dekfitu@bsuir.by.
}
% ----------- конец данных об авторах
%
% ===== Конец материалов. Все, что ниже, не изменять ! =======================================================
\end{document}