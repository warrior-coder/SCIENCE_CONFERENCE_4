\documentclass[a4paper,10pt,twoside]{article}
\usepackage{conf_template}
\setcounter{section}{0}
\setcounter{figure}{0}
\setcounter{table}{0}
\setcounter{equation}{0}
\setcounter{secnumdepth}{1}
\setcounter{secnumdepth}{1}
\begin{document}
% ================================= Начало Вашей статьи. Всё, что выше - не изменять ! ================================
%

\authors {% авторы статьи, между инициалами - пробел
А.~С.~Гудков
}%конец списка авторов
%
\topic{%начало заголовка статьи
СОВРЕМЕННЫЕ ПОДХОДЫ К СКАНИРОВАНИЮ ДОКУМЕНТОВ
}%конец заголовка статьи
%
% ------------- Аннотация -----------------------
\annotation{%% начало текста аннотации
Описывается алгоритм бесконтактного сканирования документов, устраняющий геометрические искажения, и его программная реализация на языке программирования C++ с использованием инструментария библиотеки компьютерного зрения OpenCV.
}% конец текста аннотации
\begin{multicols}{2}
% =================================== Начало основного текста ===========================================================
% ===== Первый раздел статьи =================================================
%
\section*{% заголовок первого раздела статьи
Введение}% начало текста 1 раздела статьи
Электронные форматы документов уже давно показали свои преимущества и сегодня применяются повсюду. Однако фотографии документов, получаемые на мобильных устройствах, как правило, имеют геометрические искажения, вызванные поворотом документа или наличием перспективы. В связи с этим рассмотрим алгоритм, позволяющий автоматически устранять искажения документа на изображении.
% конец текста 1 раздела статьи

% ===== 2 раздел статьи ===================================================
%
\section{% заголовок 2 раздела статьи
Алгоритм устранения искажений
}\content{% начало текста 2 раздела статьи
Процесс устранения искажений можно разделить на несколько основных этапов:

\begin{enumerate}% начало списка
\item Размытие изображения с целью удаление шумов и мелких деталей. Применяется фильтр Гаусса малой мощности 5х5.
\item Перевод цветовой модели изображения в градации серого (\textit{Grayscale}), чтобы сократить пространство поиска (Рис. 1а).
\item Вычисление границ на изображении с помощью алгоритма \textit{Canny edge detection} и их последующее расширение (Рис. 1б).
\item Нахождение множества контуров. Извлекаются контуры, составляющие отношения между границами бинарного изображения.
\item Поиск контура документа. Проводится несколько проверок. Отсеиваются контуры с площадью меньше предельной. Выполняется поиск четырехугольников, для чего производится аппроксимация контуров по алгоритму Дугласа-Пекера (Рис. 1в).
\item Расчет матрицы преобразования путем сопоставления 4-х вершин исходного контура документа с прямоугольником будущего.
\item Проекционное отображение. Проецируется изображение документа на новую плоскость просмотра путем применения матрицы преобразования к пиксельной сетке исходного изображения.
\end{enumerate}% конец списка

\image{[width=1.0\columnwidth]{GudkovAS_Images/GudkovAS_Image_01.jpg}
\caption{% Подпись под рисунком
\textit{Стадии обработки документа}
}}% конец добавления рисунка

В результате получаем выровненный документ (Рис. 1г). Для улучшения качества можно применить корректирующие цветовые фильтры ко всему документу или отдельным его частям.
}% конец текста 2 раздела статьи

\section{Выводы}
В ходе работы был рассмотрен алгоритм, устраняющий трапециевидные перспективные искажения на изображениях полученных с помощью мобильных устройств. Исходный код программы выложен на GitHub: \textcolor{blue}{\url{https://github.com/warrior-coder/OPENCV_DOCUMENT_SCANNER}}

%
% ***** Пример добавления маркированного списка ************************************
%

\references{
}\ListReferences{% начало списка литературы
\item Adrian Kaehler, Learning OpenCV 3 / Adrian Kaehler, Gary Bradski // O'Reilly Media, Inc.~-- 2016.~-- 955 P.
}% ---- конец списка литературы
%

\end{multicols}
% ====== Пример оформления данных об авторах ==============================================================
%
\authorFIO{Гудков Алексей Сергеевич}
\authorAbout{
студент 2 курса факультета информационных технологий и управления БГУИР, gudkou\_fitu@mail.ru.
}

\authorFIO{Научный руководитель: Навроцкий Анатолий Александрович}
\authorAbout{
заведующий кафедрой информационных технологий автоматизированных систем БГУИР, кандидат физико-математических наук, доцент, navrotsky@bsuir.by.
}
% ----------- конец данных об авторах
%
% ===== Конец материалов. Все, что ниже, не изменять ! =======================================================
\end{document}