\documentclass[a4paper,10pt,twoside]{article}
\usepackage{conf_template}
\setcounter{section}{0}
\setcounter{figure}{0}
\setcounter{table}{0}
\setcounter{equation}{0}
\setcounter{secnumdepth}{1}
\setcounter{secnumdepth}{1}
\begin{document}
% ================================= Начало Вашей статьи. Всё, что выше - не изменять ! ================================
%

\authors {% авторы статьи, между инициалами - пробел
С.~А.~Гусев, Н.~С.~Кучко, А.~С.~Гудков
}%конец списка авторов
%
\topic{%начало заголовка статьи
ИССЛЕДОВАНИЕ УТВЕРЖДЕНИЯ ЦЕНТРАЛЬНОЙ ПРЕДЕЛЬНОЙ ТЕОРЕМЫ С ИСПОЛЬЗОВАНИЕМ СРЕДСТВ ЯЗЫКА PYTHON
}%конец заголовка статьи
%
% ------------- Аннотация -----------------------
\annotation{%% начало текста аннотации
Исследуется утверждение центральной предельной теоремы на примере треугольного и бета- распределений с использованием языка Python и развитых возможностей Jupyter Notebook.
}% конец текста аннотации
\begin{multicols}{2}
% =================================== Начало основного текста ===========================================
%
% ===== 1 раздел статьи =================================================
%
\section*{% заголовок 1 раздела статьи
Введение}% начало текста 1 раздела статьи
Центральная предельная теорема является важной составляющей теории вероятностей и математической статистики. Но порой не всегда можно найти ее графическое представление. В связи с этим проведем исследование с целью проверки следующего  утверждения: если имеется случайная величина \textit{X} из практически любого распределения, и из этого распределения случайным образом сформирована выборка объемом \textit{N}, то выборочное среднее, определенное на основании выборки, можно приблизить нормальным распределением со средним значением, которое совпадает с математическим ожиданием исходной совокупности.
% конец текста 1 раздела статьи
%
% ===== 2 раздел статьи ===================================================
%
\section{% заголовок 2 раздела статьи
Подход к исследованию
}\content{% начало текста 2 раздела статьи
Задачей исследования является моделирование распределения выборочного среднего СВ \textit{X} при разных объемах выборок и оценка его аппроксимации с нормальной кривой. Для проведения эксперимента требуется выбрать распределение, из которого случайным образом будет формироваться выборка. Воспользуемся треугольным и бета- распределениями. Формирование выборок, подсчёт их средних, построение графиков и гистограмм осуществляется с помощью инструментария библиотек языка \textit{Python}: \textit{scipy}, \textit{numpy}, \textit{matplotlib}.
}% конец текста 2 раздела статьи
%
% ===== 3 раздел статьи ===================================================
%
\section{% заголовок 3 раздела статьи
Ход исследования
}\content{% начало текста 3 раздела статьи
Рассмотрим сперва треугольное распределение непрерывной случайной величины \textit{X}, матожидание и дисперсия которого вычисляются следующим образом:
$$
E\left(X\right)  = \frac{a+b+c}{3}.\eqno(1)
$$
$$
D\left(X\right) = \frac{a^{2}+b^{2}+c^{2}-ab-ac-bc}{18}.\eqno(2)
$$
Где $\left[a,c\right]$ -- область определения \textit{X}, $b\in\left[a,c\right]$ -- абсцисса перегиба прямой плотности вероятности. В библиотеке \textit{scipy} распределение задаётся параметром \textit{d}.
В нашем случае $d=0.2$ и \textit{X} определена на отрезке $\left[0,1\right]$, $b=0.2$.
Из данного распределения выберем 100 псевдослучайных значений. Сравним полученные результаты выборки с теоретической плотностью вероятности, график которой соответствует голубой линии (\textit{pdf -- probability distribution function}) на рисунке 1.

\image{[width=1.0\columnwidth]{GusevSA_KuchkoNS_GudkovAS_Images/GusevSA_KuchkoNS_GudkovAS_Image_01.jpg}
\caption{% Подпись под рисунком
Треугольное распределение с объемом выборки 100
}}% конец добавления рисунка

Далее, и это самое главное. При трёх и более значениях \textit{n} генерируется 1000 выборок объёма \textit{n}, вычисляется для каждой выборки среднее арифметическое. Строится гистограмма полученных выборочных средних и поверх нее накладывается график плотности соответствующего нормального распределения с параметрами:
$$
\mu = E\left(X\right) .\eqno(3)
$$
$$
\sigma = \sqrt{\frac{D(X)}{n}}.\eqno(4)
$$

Реализуем функцию \textit{build\_hist\_norm(n)} генерации нормального распределения и визуализации гистограмм по параметру объёма выборки \textit{n}. Осуществим ее вызов 6 раз со следующими значениями \textit{n}: 3, 5, 10, 50, 150, 300. Получаем результаты, представленные на рисунках 2-7.

\image{[width=8cm,height=4.2cm]{GusevSA_KuchkoNS_GudkovAS_Images/GusevSA_KuchkoNS_GudkovAS_Image_02.jpg}
\caption{% Подпись под рисунком
\textit{build\_hist\_norm(3)}
}}% конец добавления рисунка

\image{[width=8cm,height=4.2cm]{GusevSA_KuchkoNS_GudkovAS_Images/GusevSA_KuchkoNS_GudkovAS_Image_03.jpg}
\caption{% Подпись под рисунком
\textit{build\_hist\_norm(5)}
}}% конец добавления рисунка

\image{[width=8cm,height=4.2cm]{GusevSA_KuchkoNS_GudkovAS_Images/GusevSA_KuchkoNS_GudkovAS_Image_04.jpg}
\caption{% Подпись под рисунком
\textit{build\_hist\_norm(10)}
}}% конец добавления рисунка

\image{[width=8cm,height=4.2cm]{GusevSA_KuchkoNS_GudkovAS_Images/GusevSA_KuchkoNS_GudkovAS_Image_05.jpg}
\caption{% Подпись под рисунком
\textit{build\_hist\_norm(50)}
}}% конец добавления рисунка

\image{[width=8cm,height=4.2cm]{GusevSA_KuchkoNS_GudkovAS_Images/GusevSA_KuchkoNS_GudkovAS_Image_06.jpg}
\caption{% Подпись под рисунком
\textit{build\_hist\_norm(150)}
}}% конец добавления рисунка

\image{[width=8cm,height=4.2cm]{GusevSA_KuchkoNS_GudkovAS_Images/GusevSA_KuchkoNS_GudkovAS_Image_07.jpg}
\caption{% Подпись под рисунком
\textit{build\_hist\_norm(300)}
}}% конец добавления рисунка

Также рассмотрим бета-распределение со следующими числовыми характеристиками:
$$
E\left(X\right) = \frac{\alpha}{\alpha+\beta}.\eqno(5)
$$
$$
D\left(X\right) = \frac{\alpha\beta}{\left(\alpha+\beta\right)^{2}\left(\alpha+\beta+1\right)}.\eqno(6)
$$
Где $\alpha>0$, $\beta>0$. Зададим распределение с $\alpha=\beta=0.5$. Тогда функция плотности распределения и гистограмма выборки объёма 100 примут вид как на рисунке 8. Распределение выборочных средних аналогично треугольному.  

\image{[width=8cm,height=4.8cm]{GusevSA_KuchkoNS_GudkovAS_Images/GusevSA_KuchkoNS_GudkovAS_Image_08.jpg}
\caption{% Подпись под рисунком
Бета-распределение с объемом выборки 100
}}% конец добавления рисунка

}% конец текста 3 раздела статьи

\section{Выводы}
В соответствии с графическим представлением результатов хорошо прослеживается следующая закономерность: с ростом объема выборки степень аппроксимации распределения выборочных средних с нормальным распределением также растет и происходит концентрация псевдослучайных величин вокруг математического ожидания исходного распределения, что обосновывает утверждение центральной предельной теоремы.
%
% ***** Пример добавления маркированного списка ************************************
%
\references{
}\ListReferences{% начало списка литературы
\item Вентцель~Е.~В. Теория вероятностей / Е.~В.~Вентцель // Москва: «Высшая школа».~-- 2006.~-- 578 с.
}% ---- конец списка литературы
%
\end{multicols}
% ====== Пример оформления данных об авторах ==============================================================
%
\authorFIO{Гусев Станислав Александрович, Кучко Никита Сергеевич, Гудков Алексей Сергеевич}
\authorAbout{
студенты 2 курса факультета информационных технологий и управления БГУИР.
}
\authorFIO{Научный руководитель: Гуринович Алевтина Борисовна}
\authorAbout{
заместитель декана ФИТиУ, кандидат физико-математических наук, доцент, gurinovich@bsuir.by.
}
% ----------- конец данных об авторах
%
% ===== Конец материалов. Все, что ниже, не изменять ! =======================================================
\end{document}