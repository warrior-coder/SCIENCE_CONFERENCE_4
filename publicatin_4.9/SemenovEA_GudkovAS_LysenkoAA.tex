\documentclass[a4paper,10pt,twoside]{article}
\usepackage{conf_template}
\setcounter{section}{0}
\setcounter{figure}{0}
\setcounter{table}{0}
\setcounter{equation}{0}
\setcounter{secnumdepth}{1}
\setcounter{secnumdepth}{1}
\begin{document}
% ================================= Начало Вашей статьи. Всё, что выше - не изменять ! ================================
%

\authors {% авторы статьи, между инициалами - пробел
Е.~А.~Семёнов, А.~С.~Гудков, А.~А.~Лысенко
}%конец списка авторов
%
\topic{%начало заголовка статьи
СИСТЕМА ОБРАБОТКИ КОМПОНЕНТНЫХ ДАННЫХ РАСЧЁТНЫХ ЗАДАЧ И ИХ ПРЕДСТАВЛЕНИЯ В ВИДЕ ЭЛЕКТРИЧЕСКОЙ СХЕМЫ
}%конец заголовка статьи
%
% ------------- Аннотация -----------------------
\annotation{%% начало текста аннотации
Рассматривается алгоритм графической визуализации электрических цепей постоянного и переменного токов, а также разработка соответствующей программы на языке программирования C++ с использованием графического фреймворка Qt5.
}% конец текста аннотации
\begin{multicols}{2}
% =================================== Начало основного текста ===========================================================
% ===== Первый раздел статьи =================================================
%
\section*{% заголовок 1 раздела статьи
Введение}% начало текста 1 раздела статьи
В процессе изучения дисциплины «Теория электрических цепей» авторами была разработана комплексная система автоматизации дисциплины теории электрических цепей. 

В данной работе будет рассматриваться модуль графической визуализации, задачей которого является обработка компонентных данных расчётных задач и их представления в виде электрических схем, различного уровня сложности. Данный модуль основывается на результатах работы модуля синтеза расчётных задач.
% конец текста 1 раздела статьи
% ===== Второй раздел статьи ===================================================
%
\section{% заголовок 2 раздела статьи
Описание алгоритма
}\content{% начало текста 2 раздела статьи
Для успешной реализации алгоритма графической визуализации требуется представить матричные данные в математическом виде. В данной задаче математическая модель есть набор переменных, хранящих начало и конец ветви, и компонентных матриц, хранящих характеристики ветвей.

Для составления компонентных матриц требуется реализовать алгоритм обработки матричной информации из текстового документа с компонентными данными, полученного в результате выполнения программы генерации задач. На основе этого документа осуществляется преобразование матричной информации в математическую модель, т.е. заполнение переменных и компонентных матриц при помощи возможностей языка программирования \textit{C++}, а именно дополнительного потока ввода, который считывает информацию из текстового документа.

}% конец текста 2 раздела статьи
% ====== Третий раздел статьи ===================================================
%
\section{% заголовок 3 раздела статьи
Составление программы
}\content{% начало текста 3 раздела статьи
Исходные данные для графического представления хранятся в виде таблицы в текстовом документе. Реализован универсальный класс для данных, полученных из матричной информации в текстовом документе как для цепей переменного тока, так и для цепей постоянного тока. В этом классе происходит непосредственная реализация алгоритма обработки компонентных матричных данных для трансформации условия задачи в математическую модель.

Также реализован класс для графической визуализации. В нём описывается алгоритм графического представления математической модели с использованием возможностей языка программирования \textit{C++}, а также используются возможности графического фреймворка \textit{Qt5}. Данный алгоритм основывается на данных математической модели, полученной с помощью функционала, описанного в предыдущем классе. В результате выполнения функций этого класса, будет получена электрическая схема, заданного в модуле генерации уровня сложности.

Итоговый результат имеет вид схемы электрической цепи, которая сохраняется с возможностями её дальнейшего просмотра или печати.
}% конец текста 3 раздела статьи

\section{Выводы}
В результате работы авторами был рассмотрен подход к автоматизации визуализации электрических цепей с различной топологией для соответствующих расчётных задач, а также его программная реализация, позволяющая преподавателям дисциплины ТЭЦ печать читабельные схемы для студентов.
%
% ***** Пример добавления маркированного списка ************************************
%
\references{
}\ListReferences{% начало списка литературы
\item Артым~А.~Д. Новый метод расчета процессов в электрических цепях / А.~Д.~Артым, В.~А.~Филин, К.~Ж.~Есполов // СПб.: «Элмор»,~-- 2001.~-- 192 с.
\item Макс Шлее Qt 5.10. Профессиональное программирование на C++ / Макс Шлее // СПб.: «БХВ»,~-- 2018.~-- 1072 с.
}% ---- конец списка литературы
%

\end{multicols}
% ====== Пример оформления данных об авторах ==============================================================
%

\authorFIO{Семёнов Егор Александрович}
\authorAbout{
студент 2 курса кафедры информационных технологий автоматизированных систем БГУИР, egor123semenov@gmail.com.
}

\authorFIO{Гудков Алексей Сергеевич}
\authorAbout{
студент 2 курса кафедры информационных технологий автоматизированных систем БГУИР, gudkov\_fitu@mail.ru.
}

\authorFIO{Лысенко Антон Александрович}
\authorAbout{
студент 2 курса кафедры информационных технологий автоматизированных систем БГУИР, toshka.lysenko.15@gmail.com.
}

\authorFIO{Научный руководитель: Шилин Леонид Юрьевич}
\authorAbout{
декан факультета информационных технологий и управления БГУИР, доктор технических наук, профессор, dekfitu@bsuir.by.
}
% ----------- конец данных об авторах
%
% ===== Конец материалов. Все, что ниже, не изменять ! =======================================================
\end{document}