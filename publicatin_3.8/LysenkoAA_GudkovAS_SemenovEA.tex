\documentclass[a4paper,10pt,twoside]{article}
\usepackage{conf_template}
\setcounter{section}{0}
\setcounter{figure}{0}
\setcounter{table}{0}
\setcounter{equation}{0}
\setcounter{secnumdepth}{1}
\setcounter{secnumdepth}{1}
\begin{document}
% ================================= Начало Вашей статьи. Всё, что выше - не изменять ! ================================
%

\authors {% авторы статьи, между инициалами - пробел
А.~А.~Лысенко, А.~С.~Гудков, Е.~А.~Семёнов
}%конец списка авторов
%
\topic{%начало заголовка статьи
АВТОМАТИЗИРОВАННАЯ СИСТЕМА СИНТЕЗА ЭЛЕКТРИЧЕСКИХ ЦЕПЕЙ ДЛЯ УЧЕБНЫХ ЗАДАЧ
}%конец заголовка статьи
%
% ------------- Аннотация -----------------------
\annotation{%% начало текста аннотации
В статье описывается реализация алгоритма генерирования задач по теории электрических цепей для постоянного и переменного тока в форме табличных данных по заданным условиям.
}% конец текста аннотации
\begin{multicols}{2}
% =================================== Начало основного текста ===========================================================
% ===== Первый раздел статьи =================================================
%
\section*{% заголовок 1 раздела статьи
Введение}% начало текста 1 раздела статьи
В процессе изучения дисциплины «Теория электрических цепей» у нас возникали трудности с поиском задач на определенную тематику для тренировки их решения, а также у преподавателей для проверки знаний студентов. Было принято решение написать программное обеспечение для генерирования задач в форме табличных данных, на заданную тематику. 
% конец текста 1 раздела статьи
% ===== Второй раздел статьи ===================================================
%
\section{% заголовок 2 раздела статьи
Предварительные условия
}\content{% начало текста 2 раздела статьи
Для удобства генерирования представим электрическую цепь в виде ориентированного графа, сформированного по определенным правилам, направления ветвей которого совпадают с направлениями токов в ветвях, а вершины с узлами цепи. В нашей программе необходимо задать некоторые начальные условия такие как:

\textbf{--} диапазон значений для сопротивления резисторов;

\textbf{--} диапазон значений для источника тока и источника напряжения;

\textbf{--} количество(диапазон) источников тока и источников напряжения;

\textbf{--} количество(диапазон) сопротивлений;

\textbf{--} количество уравнений для решения методом контурных токов;

\textbf{--} количество уравнений для решения методом узловых потенциалов.

}% конец текста 2 раздела статьи
% ====== Третий раздел статьи ===================================================
%
\section{% заголовок 3 раздела статьи
Алгоритм генерирования
}\content{% начало текста 3 раздела статьи
Исходя из метода расчета и количества уравнений для решения синтезируется граф цепи. Далее из предварительных условий генерируются методом случайных величин значения сопротивления. Аналогичные действия производим для источника напряжения и источника тока. Также нам нужно узнать количество ветвей в цепи исходя из количества источников напряжения и источников тока. После чего методом случайных величин расставляем исходные элементы цепи в произвольной форме с соблюдением определенных правил.

В результате получаем модель электрической цепи, представленной в виде ориентированного графа. По ее графу получаем электрическую цепь, которая представлена в текстовом формате в виде таблицы. В таблице указаны следующие параметры цепи: 

\textbf{--} количество ветвей;

\textbf{--} направление токов в ветви в формате начальный узел и конечный узел;

\textbf{--} сопротивления резисторов;

\textbf{--} значения источников ЭДС;

\textbf{--} значения источников тока.

При генерировании электрической цепи для переменного тока таблица дополняется следующими параметрами:

\textbf{--} реактивные сопротивления конденсаторов;

\textbf{--} реактивные сопротивления катушек индуктивности;

\textbf{--} значения источников тока в комплексной форме;

\textbf{--} значения источников напряжения в комплексной форме.

}% конец текста 3 раздела статьи

\section{Выводы}
В ходе работы был рассмотрен подход к генерированию электрических цепей постоянного и переменного тока на основе предварительных условий. Данное программное обеспечение полезно студентам для отработки навыков решения задач, а также и преподавателям для составления задач на определенную тематику с большим количеством вариантов.

%
% ***** Пример добавления маркированного списка ************************************
%
\references{
}\ListReferences{% начало списка литературы
\item Атабеов~Г.~И. Теоретические основы электротехники / Атабеов~Г.~И. // М.: «Энергия»,~-- 1978.~-- 592 с.
\item Харари~Ф. Теория Графов / Харари~Ф. // М.: «Мир»,~-- 1973.~-- 300 с.
}% ---- конец списка литературы
%

\end{multicols}
% ====== Пример оформления данных об авторах ==============================================================
%

\authorFIO{Лысенко Антон Александрович}
\authorAbout{
студент 2 курса кафедры информационных технологий автоматизированных систем БГУИР, toshka.lysenko.15@gmail.com.
}

\authorFIO{Гудков Алексей Сергеевич}
\authorAbout{
студент 2 курса кафедры информационных технологий автоматизированных систем БГУИР, gudkov\_fitu@mail.ru.
}

\authorFIO{Семёнов Егор Александрович}
\authorAbout{
студент 2 курса кафедры информационных технологий автоматизированных систем БГУИР, egor123semenov@gmail.com.
}

\authorFIO{Научный руководитель: Шилин Леонид Юрьевич}
\authorAbout{
декан факультета информационных технологий и управления БГУИР, доктор технических наук, профессор, dekfitu@bsuir.by.
}
% ----------- конец данных об авторах
%
% ===== Конец материалов. Все, что ниже, не изменять ! =======================================================
\end{document}