\documentclass[a4paper,10pt,twoside]{article}
\usepackage{conf_template}
\setcounter{section}{0}
\setcounter{figure}{0}
\setcounter{table}{0}
\setcounter{equation}{0}
\setcounter{secnumdepth}{1}
\setcounter{secnumdepth}{1}
\begin{document}
% ================================= Начало Вашей статьи. Всё, что выше - не изменять ! ================================
%

\authors {% авторы статьи, между инициалами - пробел
Н.~С.~Кучко, С.~А.~Гусев, А.~С.~Гудков
}%конец списка авторов
%
\topic{%начало заголовка статьи
ПРИМЕНЕНИЕ ЗАКОНА РАСПРЕДЕЛЕНИЯ ПРИ РЕШЕНИИ ЗАДАЧ НА ПРОСТЕЙШИЙ ПУАССОНОВСКИЙ ПОТОК
}%конец заголовка статьи
%
% ------------- Аннотация -----------------------
\annotation{%% начало текста аннотации
Рассматриваются применение закона распределения при решении задач на простейший пуассоновский поток, в частности на парадокс времени ожидания. Рассматриваемые задачи являются прекрасным примером для понимания данной темы.
}% конец текста аннотации
\begin{multicols}{2}
% =================================== Начало основного текста ===========================================================
% ===== Первый раздел статьи =================================================
%
\section*{% заголовок 1 раздела статьи
Введение}% начало текста 1 раздела статьи
Каждый наверняка встречался с такой ситуацией: вы приходите на остановку. Написано, что автобус ходит каждые 10 минут. Засекаете время... Однако автобус приходит только через 11 минут. По идее, если автобусы приходят каждые 10 минут, а вы придёте в случайное время, то среднее ожидание должно составлять около 5 минут. Но в действительности автобусы не прибывают точно по расписанию, поэтому вы можете ждать дольше. Оказывается, при некоторых разумных предположениях можно прийти к поразительным выводам.
% конец текста 1 раздела статьи

% ===== 2 раздел статьи ===================================================
%
\section{% заголовок 2 раздела статьи
Ход исследования
}\content{% начало текста 2 раздела статьи
Парадокс времени ожидания является частным случаем более общего явления -- парадокса инспекции, который возникает всякий раз, когда вероятность наблюдения количества связана с наблюдаемым количеством. Например, анкетирование студентов университета о среднем размере группы. Хотя школа правдиво говорит о среднем количестве 30 студентов в группе, но средний размер группы с точки зрения студентов гораздо больше. Причина в том, что в больших группах больше студентов, что и выявляется при их опросе.

В случае автобусного графика с заявленным 10-минутным интервалом иногда промежуток между прибытиями длиннее 10 минут, а иногда и короче. И если придти на остановку в случайное время, то у вас больше вероятность столкнуться с более длинным интервалом, чем с более коротким. И поэтому логично, что средний промежуток времени, между интервалами ожидания дольше, чем средний промежуток времени между автобусами, потому что более длинные интервалы чаще встречаются в выборке. Но парадокс времени ожидания делает более сильное заявление: если средний интервал между автобусами составляет N минут, то среднее время ожидания для пассажиров составляет 2N минут.

Об этой проблеме можно рассуждать так: процесс Пуассона -- это процесс без памяти (т. е. история событий не имеет никакого отношения к ожидаемому времени следующего события). Поэтому по приходу на автобусную остановку среднее время ожидания автобуса всегда одинаково (в нашем случае -- 10 минут), независимо от того, сколько времени прошло с момента предыдущего автобуса. При этом не имеет значения, как долго вы уже ждали: ожидаемое время до следующего автобуса всегда ровно 10 минут: в пуассоновском процессе вы не получаете «кредит» за время, проведённое в ожидании.
}% конец текста 2 раздела статьи

\section{Выводы}
Парадокс времени ожидания является интересной отправной точкой для обсуждений, которые включают в себя моделирование, теорию вероятности и сравнение статистических предположений с реальностью. Хотя было установлено, что в реальном мире автобусные маршруты подчиняются некоторой разновидности парадокса инспекции, приведённый в ходе работы анализ довольно убедительно показывает: основные предположения, лежащие в основе рассматриваемого парадокса. На самом деле в хорошо управляемой системе общественного транспорта есть специально структурированные расписания, чтобы избежать такого поведения: автобусы не начинают свои маршруты в случайное время в течение дня, а стартуют по расписанию, выбранному для наиболее эффективной перевозки пассажиров.

\references{
}\ListReferences{% начало списка литературы
\item Гмурман~В.~Е. Теория вероятностей и математическая статистика / В.~Е.~Гмурман // Москва: «Высшая школа»,~-- 2003.~-- 480 с.
}% ---- конец списка литературы
%

\end{multicols}
% ====== Пример оформления данных об авторах ==============================================================
%
\authorFIO{Кучко Никита Сергеевич}
\authorAbout{
студент 2 курса кафедры информационных технологий автоматизированных систем БГУИР, nikitakuchko2002@mail.ru.
}

\authorFIO{Гусев Станислав Александрович}
\authorAbout{
студент 2 курса кафедры информационных технологий автоматизированных систем БГУИР, st.al.gusev@gmail.com.
}

\authorFIO{Гудков Алексей Сергеевич}
\authorAbout{
студент 2 курса кафедры информационных технологий автоматизированных систем БГУИР, gudkou\_fitu@mail.ru.
}

\authorFIO{Научный руководитель: Гуринович Алевтина Борисовна}
\authorAbout{
заместитель декана ФИТиУ, кандидат физико-математических наук, доцент, gurinovich@bsuir.by.
}
% ----------- конец данных об авторах
%
% ===== Конец материалов. Все, что ниже, не изменять ! =======================================================
\end{document}